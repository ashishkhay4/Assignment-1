\documentclass[12pt]{article}

\usepackage{graphicx}
\graphicspath{{folder a/}}

\title{\textbf{National Instituate Of  Technology  Raipur}\ }
\author{\textbf{Name : ASHISH}\\ Roll.no : 21111013\\Branch : Biomedical Enginering \\}


\begin{document}
\maketitle
\begin{figure}[h]
\centering
\includegraphics[scale=0.7]{nit.png}

\end{figure}

\textbf{ASSIGNMENT -1  : ANATOMY AND PHYSIOLOGY  }\\\\

\textbf {UNDER THE SUPERVISION : SAURABH GUPTA SIR }\\\\

\textbf{       BIOMEDICAL DEPARTMENT}




\section{What is  Anatomy and Physiology ?}

Anatomy and physiology, which study the structure and function of organisms and their parts respectively, make a natural pair of related disciplines, and are often studied together. Human anatomy is one of the essential basic sciences that are applied in medicine.


Anatomy is the study of the structures associated with the human body. Physiology is the study of the function of each of these structures. The human body is often thought of as a complicated machine.

\begin{figure}[h]
\centering
\includegraphics[scale=0.3]{nitt.jpg}

\end{figure}

\section{What is Levels of Structural Organization and Body Systems ?}
Name the six levels of organization of the human body. Chemical, cellular, tissue, organ, organ system, organism.


Higher levels of organization are built from lower levels. Therefore, molecules combine to form cells, cells combine to form tissues, tissues combine to form organs, organs combine to form organ systems, and organ systems combine to form organisms. 

\section{ What is Characteristics of the Living Human Organism ?}
\subsection{The seven characteristics of life include:}
 .responsiveness to the environment;\\
.growth and change;\\
.ability to reproduce;\\
.have a metabolism and breathe;\\
.maintain homeostasis;\\
.being made of cells; and.\\
passing traits onto offspring.

\section{ Homeostasis: } 
Homeostasis is any self-regulating process by which an organism tends to maintain stability while adjusting to conditions that are best for its survival. If homeostasis is successful, life continues; if it's unsuccessful, it results in a disaster or death of the organism.
\section{ Basic Anatomical Terminology:}
Anatomical terminology is a form of scientific terminology used by anatomists, zoologists, and health professionals such as doctors.To compare the location of body parts relative to each other, anatomy uses some universal directional terms: anterior, posterior, ventral, dorsal, distal.
\section{ What is  Medical Imaging ?}
Medical imaging refers to several different technologies that are used to view the human body in order to diagnose, monitor, or treat medical conditions.

 Medical imaging also establishes a database of normal anatomy and physiology to make it possible to identify abnormalities. Although imaging of removed organs and tissues can be performed for medical reasons, such procedures are usually considered part of pathology instead of medical imaging.
\end{document}